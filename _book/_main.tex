% Options for packages loaded elsewhere
\PassOptionsToPackage{unicode}{hyperref}
\PassOptionsToPackage{hyphens}{url}
%
\documentclass[
]{article}
\usepackage{amsmath,amssymb}
\usepackage{lmodern}
\usepackage{iftex}
\ifPDFTeX
  \usepackage[T1]{fontenc}
  \usepackage[utf8]{inputenc}
  \usepackage{textcomp} % provide euro and other symbols
\else % if luatex or xetex
  \usepackage{unicode-math}
  \defaultfontfeatures{Scale=MatchLowercase}
  \defaultfontfeatures[\rmfamily]{Ligatures=TeX,Scale=1}
\fi
% Use upquote if available, for straight quotes in verbatim environments
\IfFileExists{upquote.sty}{\usepackage{upquote}}{}
\IfFileExists{microtype.sty}{% use microtype if available
  \usepackage[]{microtype}
  \UseMicrotypeSet[protrusion]{basicmath} % disable protrusion for tt fonts
}{}
\makeatletter
\@ifundefined{KOMAClassName}{% if non-KOMA class
  \IfFileExists{parskip.sty}{%
    \usepackage{parskip}
  }{% else
    \setlength{\parindent}{0pt}
    \setlength{\parskip}{6pt plus 2pt minus 1pt}}
}{% if KOMA class
  \KOMAoptions{parskip=half}}
\makeatother
\usepackage{xcolor}
\usepackage[margin=1in]{geometry}
\usepackage{longtable,booktabs,array}
\usepackage{calc} % for calculating minipage widths
% Correct order of tables after \paragraph or \subparagraph
\usepackage{etoolbox}
\makeatletter
\patchcmd\longtable{\par}{\if@noskipsec\mbox{}\fi\par}{}{}
\makeatother
% Allow footnotes in longtable head/foot
\IfFileExists{footnotehyper.sty}{\usepackage{footnotehyper}}{\usepackage{footnote}}
\makesavenoteenv{longtable}
\usepackage{graphicx}
\makeatletter
\def\maxwidth{\ifdim\Gin@nat@width>\linewidth\linewidth\else\Gin@nat@width\fi}
\def\maxheight{\ifdim\Gin@nat@height>\textheight\textheight\else\Gin@nat@height\fi}
\makeatother
% Scale images if necessary, so that they will not overflow the page
% margins by default, and it is still possible to overwrite the defaults
% using explicit options in \includegraphics[width, height, ...]{}
\setkeys{Gin}{width=\maxwidth,height=\maxheight,keepaspectratio}
% Set default figure placement to htbp
\makeatletter
\def\fps@figure{htbp}
\makeatother
\setlength{\emergencystretch}{3em} % prevent overfull lines
\providecommand{\tightlist}{%
  \setlength{\itemsep}{0pt}\setlength{\parskip}{0pt}}
\setcounter{secnumdepth}{5}
\usepackage{booktabs}
\ifLuaTeX
  \usepackage{selnolig}  % disable illegal ligatures
\fi
\usepackage[]{natbib}
\bibliographystyle{plainnat}
\IfFileExists{bookmark.sty}{\usepackage{bookmark}}{\usepackage{hyperref}}
\IfFileExists{xurl.sty}{\usepackage{xurl}}{} % add URL line breaks if available
\urlstyle{same} % disable monospaced font for URLs
\hypersetup{
  pdftitle={Rにる作業の自動化・効率化 -各種パッケージの活用方法-},
  pdfauthor={Toshikazu Masumura},
  hidelinks,
  pdfcreator={LaTeX via pandoc}}

\title{Rにる作業の自動化・効率化 -各種パッケージの活用方法-}
\author{Toshikazu Masumura}
\date{2023-04-26}

\begin{document}
\maketitle

{
\setcounter{tocdepth}{2}
\tableofcontents
}
\hypertarget{ux306fux3058ux3081ux306b}{%
\section*{はじめに}\label{ux306fux3058ux3081ux306b}}
\addcontentsline{toc}{section}{はじめに}

誰でもそうだろうが,面倒くさい仕事はしたくない.
というか,したくないことが面倒くさいのだろう.
ニワトリかタマゴのような話は別として,できることなら,面倒な作業は自動化したい.
もちろんすべての仕事を自動化できるわけでもないし,文章執筆のように作業内容によっては自動化すべきでないこともある.

作業の自動化には,プログラミング言語を使うことが多い.
自動化でよく使われる言語としては,Pythonがある.
Pythonは比較的習得しやすい言語らしく,多くの人が使っている.
自分自身も多少はPythonを使えるものの,それよりもRの方が慣れている.
できることなら(ほぼ)全ての作業をRでやってしまいたい.
そんなわけで,この文章ではRを使った作業の自動化や効率化方法を紹介する.

基本的に独学でここまで来たので,我流のスクリプトや汚いコードが多くあると思われるがご容赦頂きたい.
また,改善案をご教示いただければありがたい.

\href{mailto:matutosi@gmail.com}{\nolinkurl{matutosi@gmail.com}}

\hypertarget{R}{%
\section{Rとは}\label{R}}

Rは,統計解析環境であるとともに,プログラミング言語である.
プログラミング言語としては,やや特殊な文法をもっている.
そのため,他の言語よりも好き嫌いが激しいと思われる.

\hypertarget{rux306eux7279ux5fb4}{%
\subsection{Rの特徴}\label{rux306eux7279ux5fb4}}

文法的に特殊な点では,代入での「\textless-」使用やパイプ(「\%\textgreater\%」や「\textbar\textgreater」(R-4.1以降))を多様することが挙げられる.

他の多くのプログラミング言語では,代入には「=」を使用する.
Rでも「=」を使えるが,「\textless-」を好んで使う人が多いと思われる.
少なくとも私はそうしている.
理由を問われても特には思いつかないが,慣れていることや,コードを見た時にすぐにRだとわかるぐらいだろうか.
実用的には,「=」を入力するよりも手間がかかるし面倒なはずであるが,すでに手が慣れてしまっている.

パイプを最初に見たときには違和感を覚えたが,使い始めるとクセになる.
クセになるだけではなく,同じ変数名を何度も使ったり,中間の変数名を考えなくて良い点で優れている.
第1引数を省略できるため,入力の手間が少ない.
パイプだけの恩恵ではなく,tidyverseの利用も大きいが,コードが簡潔になって,コードの使い回しもし易い.
パイプにはこのような多くの利点がある.

他にも「::」がやたらと出てくることや,実行速度が遅いなど欠点もそれなりにある.
そもそも完璧なプログラミング言語などそ存在せず,それぞれが利点・欠点を持っている.
それぞれの得意な分野でうまく使うことが重要である.
とはいいながら,多くのプログラミング言語を習得するのは困難である.
私はこれまでちょっとだけでもかじったことのある言語としては,
FORTRAN,Perl,Ruby,C,C++,VBA,Java,Python,JavaScript,Rなどがある.
それぞれなんとなく読むことはできるが,実際によく使うのはRだけである.
JavaScriptはその次に使っているが,頻度は非常に低い.
Pythonは勉強中である.

\hypertarget{ux70b9ux7a81ux7834}{%
\subsection{1点突破}\label{ux70b9ux7a81ux7834}}

プログラミング言語にはそれぞれ得意分野があることは確かだが,垣根を超えて使うことはできる.
例えば,RからPythonを使うパッケージとしてreticulateがあり,PythonからRを使うパッケージとしてrpy2がある.
つまり,1つのプログラミング言語でしか実行できないものはほとんどなく,使いたい言語を使って勉強したい言語を勉強すれば良い.

Rの総本山であるCRANには,他にも様々なパッケージがあり,いろいろな道具が揃っている.
多くの言語を習得するのも良いが,いっそのこと1つの言語をある程度極めて,そこから使えるものは使うのも良い方法と言えるだろう.
つまり一点突破の手法である.
そこで,Rのパッケージを使って,各種操作をすることを目的にこの文章を執筆した(している).

\hypertarget{ux30d8ux30ebux30d7ux30c9ux30adux30e5ux30e1ux30f3ux30c8}{%
\subsection{ヘルプドキュメント}\label{ux30d8ux30ebux30d7ux30c9ux30adux30e5ux30e1ux30f3ux30c8}}

ところで,Rにはヘルプ・ドキュメントがしっかりしているというのも非常に良い.
ヘルプは,「?関数名」としてRから直接呼び出すことができ,関数の引数,返り値,使用例などが詳しく解説されていることが多い.
ユーザーとしてはいちいちネットや書籍で調べなくても良いのが心強い.
パッケージの開発者としては,既存のパッケージのドキュメントがしっかりしているため,それに合わせるべくしっかりとしたドキュメントを書かなければならないという圧力はあることは事実である.
ただ,ドキュメントをしっかり作っておかないと,開発者も関数の使い方を忘れてしまうことになりかねないため,結局は「他人のためならず」と言えるだろう.

\end{document}
